% # 公立はこだて未来大学・修士論文書式定義ファイル
%
% (https://github.com/kmiya/naist-thesis-tmpl を一部参照)
% 
% ## 改訂履歴:
% - 2019/11/18 初版 作成者:三上貞芳

% ## 使用法;
% - main.texを参照してください.
% - **このファイルを変更する必要はありません**

\usepackage[dvipdfmx]{graphicx}
\usepackage[utf8]{inputenc}
\usepackage[T1]{fontenc}
\usepackage{lmodern}
\usepackage{amsmath,amssymb,amsthm}
\let\equation\gather
\let\endequation\endgather
\usepackage{fancybox}
\usepackage[flushmargin,symbol]{footmisc}
\usepackage[nottoc]{tocbibind}
\usepackage[dvipdfmx,%
 bookmarks=true,%
 bookmarksnumbered=true,%
 setpagesize=false,%
 colorlinks=false,%
 linkbordercolor={0.8 0.8 0.8},%
 citebordercolor={0.8 0.8 0.8},%
 pdfborder={0 0 0.6},%
% urlcolor=black,linkcolor=black,citecolor=black,%
 pdftitle={},% 修論のタイトルを入れる
 pdfauthor={},% 名前を入れる
 pdfsubject={Master's thesis},%
 pdfkeywords={\ekeywords}]{hyperref}
\usepackage{pxjahyper}

% ページレイアウト
\textheight=20.6truecm          % 縦
\textwidth=14.5truecm           % 横
\oddsidemargin=0.6truecm        % 左マージン(1inオフセット後)
\evensidemargin=-3.8truecm      % 右マージン(1inオフセット後)

% フォント等調整
% 参考文献
\def\thebibliography#1{\chapter*{参考文献\markboth
 {参 考 文 献}{参 考 文 献}\addcontentsline{toc}{chapter}{参考文献}}\list
 {[\arabic{enumi}]}{\settowidth\labelwidth{[#1]}\leftmargin\labelwidth
 \advance\leftmargin\labelsep
 \usecounter{enumi}}
 \def\newblock{\hskip .11trueem plus .33trueem minus -.07trueem}
 \sloppy
 \sfcode`\.=1000\relax}
\let\endthebibliography=\endlist

% 章
\makeatletter%%
\def\@makechapterhead#1{\hbox{}%
  \vskip-1\Cvs
  {\parindent\z@
%  \reset@font\LARGE\bfseries
   \raggedright\reset@font\Large\bfseries% 左揃え
   \ifnum \c@secnumdepth >\m@ne
     \setlength\@tempdima{\linewidth}%
     \vtop{\hsize\@tempdima%
         \@chapapp\thechapter\@chappos\mbox{\ \ }%
     #1}%
   \else
     #1\relax
   \fi}\nobreak\vskip1\Cvs}
\makeatother%%

\makeatletter%%
\def\@makeschapterhead#1{\hbox{}%
  \vskip-1\Cvs
  {\parindent \z@ \raggedright
    \normalfont
    \interlinepenalty\@M
    \Large\headfont #1\par\nobreak
    \vskip1\Cvs}}
\makeatother%%

% 節
\makeatletter%%
\renewcommand{\section}{%
  \@startsection{section}% #1 見出し
   {1}% #2 見出しのレベル
   {\z@}% #3 横組みの場合,見出し左の空き(インデント量)
   {1.5\Cvs \@plus.5\Cdp \@minus.2\Cdp}% #4 見出し上の空き
   {.5\Cvs \@plus.3\Cdp}% #5 見出し下の空き (負の値なら見出し後の空き)
  {\raggedright\reset@font\large\bfseries}% 左揃え
}%
\makeatother%%

% 小節
\makeatletter%%
\renewcommand{\subsection}{%
  \@startsection{subsection}% #1 見出し
   {1}% #2 見出しのレベル
   {\z@}% #3 横組みの場合,見出し左の空き(インデント量)
   {1.5\Cvs \@plus.5\Cdp \@minus.2\Cdp}% #4 見出し上の空き
   {.5\Cvs \@plus.3\Cdp}% #5 見出し下の空き (負の値なら見出し後の空き)
  {\raggedright\reset@font\normalsize\bfseries}% 左揃え
}%
\makeatother%%

% 表題
\makeatletter
\def\@startsection#1#2#3#4#5#6{%
  \if@noskipsec \leavevmode \fi
  \par
  \@tempskipa #4\relax
  \if@english \@afterindentfalse \else \@afterindenttrue \fi
  \ifdim \@tempskipa <\z@
    \@tempskipa -\@tempskipa \@afterindentfalse
  \fi
  \if@nobreak
    \everypar{}%
  \else
    \addpenalty\@secpenalty
    \ifdim \@tempskipa >\z@
      \vskip\@tempskipa
      \if@slide\else
        \null
        \vspace{-\baselineskip}%
      \fi
    \fi
  \fi
  \noindent
  \@ifstar
    {\@ssect{#3}{#4}{#5}{#6}}%
    {\@dblarg{\@sect{#1}{#2}{#3}{#4}{#5}{#6}}}}
\makeatother

% 式番号
\makeatletter
  \renewcommand{\theequation}{%
  \thesection.\arabic{equation}}
    \@addtoreset{equation}{section}
\makeatother


% 図番号
\makeatletter
 \renewcommand{\thefigure}{%
  \thechapter.\arabic{figure}}
   \@addtoreset{figure}{chapter}
 \makeatother
\makeatletter

% 目次に小節を表示
\setcounter{tocdepth}{4}

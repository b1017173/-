\chapter{対話雰囲気推定モデル\label{sec:estimation_model}}
\thispagestyle{plain}

本章では第\ref{node:estimation_module}節で触れた対話雰囲気推定モデルの詳細について述べる.

\section{対話雰囲気モデルの構築\label{node:develop_estimation_model}}

本研究では,通話中の音声から機械学習によって対話雰囲気を推定する.
豊田らは二者対話を対象に発話時間特徴を特徴量とする機械学習を用いた対話雰囲気推定を行なっている.
豊田らの構築したモデルは「盛り上がり」「まじめさ」「噛み合い」「明るさ」「親密さ」「対等さ」の6つの雰囲気を推定対象としており,それぞれ肯定,否定の2値で分類している.
特に「盛り上がり」「まじめさ」「親密さ」の推定においては全体正答率が80%を超える高い値を示していることから,本研究でもこの手法に基づいてモデルの構築を行う.
モデルの構築手順を以下に示す.

\subsection{学習データの収集}

はじめに,学習データの収集を行う.
豊田らは音声コーパス\cite{PASD}を用いているのに対し,本研究では実際の作業通話の録音を採用する.
これは一般的な対話と作業通話中の対話は性質が異なるためである.
例えば,一般的な対話では沈黙は気まずく回避される傾向があるが,作業通話における沈黙は作業に集中していることの現れであり許容される傾向がある.
また,作業通話中の対話においては突発的な話題の変化が起こりやすく,独り言が行き交うことが多いなどの特徴がある.
このような作業通話中の対話の独特な性質が,対話雰囲気の形成に影響を与えていると十分に考えられることから,本研究では実際の作業通話の音声を学習に用いる.
学習データとして1~4名の話者からなる計6時間弱の作業通話音声の収集を行なった.
各対話の記録時間は10~120秒である.そして,各抽出データに教師ラベルを付与する.
教師ラベルは1つの抽出データに対して,推定対象の雰囲気ごとに肯定,否定のどちらかの値を与える.
本研究の推定対象雰囲気は「盛り上がり」,「真面目さ」,「親密さ」,「くつろぎ」の4つとする.
「盛り上がり」は対話の活発性や参加者の対話への積極性を指す.
「真面目さ」は対話の生産性や参加者が対話にどのくらい熱心に取り組んでいるかを指す.
「明るさ」は対話の健全さや参加者の性格を指す.
「くつろぎ」は参加への心理的安全性や対話への満足感を指す.以上の手法により,本研究では合計130対話を抽出した.

\subsection{特徴量の抽出}

各対話データから特徴量抽出のための発話集合の抽出を行う.
発話集合とは話者ごとの発話時間を記録した集合のことを指す.
また,単独発話と同時発話それぞれの発話集合を別に記録する.
以上は豊田らの手法と同様であるが,本研究ではそれに加えて全話者の発話集合を集約した全発話集合及び,全発話集合に非発話集合を加えた全集合も生成する.

生成した発話集合から特徴量の抽出を行う.
豊田らは各集合に対する統計量とその統計量を比較した値(以下,「比較量」)を算出し特徴量としている.
本研究でも統計量と比較量を特徴量とするが,具体的な算出式は異なる.
まず,それぞれの集合に対して平均値,標準偏差,最大値,要素数を算出する.
加えて,全集合以外の集合に対して占有率を算出し,全発話集合と全集合に対して合計値を算出する.
以上全てを特徴量として用いる.その後それぞれの特徴量の比較を行いそれらも特徴量とする.
これは各話者単体や全体に着目するだけでなく,各話者間や各話者と全体の比較を行うことで一部の雰囲気の推定に有効と仮説を立てたためである.
例えば,話者間に極端な発話時間・回数の差があるデータと,話者間の発話時間・回数に差があまりないデータが存在した場合,前者に比べ後者は議論が活発化しており真面目な雰囲気なことが多いのではないかと推測できる.
比較量を求める際に比較する値はそれぞれの集合の同じ統計量である.
例えば,発話時間が最も長いA話者の単独発話集合とB話者の単独発話集合を平均値で比較する場合は,A話者の単独発話集合の平均値をB話者の単独発話集合の平均値で除算することで算出する.
比較の際には話者間発話比較,話者間同時発話比較,話者内発話・非発話比較,全発話・発話比較を行う.
結果として各対話データにつき210個の特徴量が算出される.
ただし,210個というのは最大値であり実際に算出される特徴量数は話者数により異なる.
そのため,前処理として学習データ間に利用特徴量数の差があった場合は,話者数に合わせ特徴量や学習データの削除を行い新しい学習データとする.

\subsection{学習と特徴量選択}

算出された特徴量を用いて学習を行う.
本研究の学習の方法は,対象とする雰囲気や話者数が豊田らと異なるため,一部独自の手法を採用する.
本研究では,雰囲気,話者数ごとに対話雰囲気推定モデルの構築を行う.
これは各雰囲気や話者数によって有効な特徴量が異なるという仮説に基づいているためである.
本研究で学習を行う際には対象雰囲気,対象話者数,学習アルゴリズムの設定を行う.
本研究の対象雰囲気は「盛り上がり」,「真面目さ」,「明るさ」,「くつろぎ」の4つから選択する.
対象話者数は2〜4名から選択する.
学習モデルにはどの学習モデルが本研究の学習において有効であるか比較を行うため,Linear SVC,k近傍法,SVC,Naïve Bayesの4つを採用する.

また,本研究では豊田らの手法にならい,遺伝的アルゴリズム(以下,「GA」)を用いた特徴量選択を行う.
特徴量選択を行うことで精度の向上や安定化が期待できる.
具体的な手法も豊田らの手法を参考にする.
GAでは特徴量数と同じ長さのビット列からなる染色体を生成し,各特徴量の利用有無を1:有効,0:無効で表現する.
GAにおける設定の一部を表\ref{tab:ga_setting}に記載する.
表中の評価における「選択特徴量数と正答率の重み付け和による評価」は以下の式によって算出する.
式中の$W$は正答率と選択特徴量数の重み付けを表現しており,$[0, 1]$の値をとる.

\begin{table}[t]
    \caption{遺伝的アルゴリズムの設定}
    \centering
    \begin{tabular}{ll}
        \hline
        設定項目 & 範囲・候補 \\
        \hline\hline
        評価 & 正答率による評価 \\
        & 選択特徴量数と正答率の重み付け和による評価 \\
        \hline
        検証 & 交差検証の有無 \\
        \hline
        突然変異確率 & 突然変異確率 \\
        \hline
        交叉  & 一様交叉 \\
        & 二点交叉 \\
        \hline
        集団の大きさ  & 100 \\
        \hline
        エリート染色体選択数  & 20 \\
        \hline
        世代数  & 250 \\
        \hline
    \end{tabular}
    \label{tab:ga_setting}
\end{table}

\begin{equation}
    評価値 = \frac{正答数}{検証データ数} W + (1 - \frac{選択特徴量数}{全特徴量数}) (1 - W)
\end{equation}

\section{対話雰囲気モデルの評価と考察}

第\ref{node:develop_estimation_model}節で述べた手法に基づいて対話雰囲気推定モデルを構築した結果を表\ref{tab:learn_result_with_ga}に示す.
表\ref{tab:learn_result_with_ga}には各雰囲気について各話者数のモデルの学習結果の一例を掲載している.
表\ref{tab:learn_result_with_ga}に掲載している学習結果のモデルは全て特徴量選択を行ったモデルである.
表中のモデルは表\ref{tab:ga_setting}のうち評価は選択特徴量数と正答率の重み付け和による評価を採用し,突然変異率は個体突然変異率0.05,遺伝子突然変異率0.10,交叉方法には一様交叉を採用したモデルである.
評価における重み付け$W$は0.50に設定している.
また,学習アルゴリズムにはNaïve Bayesを採用している.
特徴量選択を行う前のモデルの正答率は0.7462であり,特徴量数は144個であった(「盛り上がり」を対象とした話者数3名のモデル).
同雰囲気,話者数を対象としたモデルと比較すると,特徴量選択を行うことで正答率の向上と選択特徴量数の減少が確認できる(正答率0.7956,選択特徴量数21).
加えて,雰囲気や話者数の違いによって精度に大きな差が出ていないことがわかる.
しかし,何度か学習を繰り返すと話者数の増加に伴って正答率が0.40や0.98など極端な数値になることがある.
これは学習データ数が少ないことが原因と考えられる.
例えば,検証に利用できるデータ数が2つしかない場合は正答率が0.0,0.5,1.0の3つしか取り得ない.
このように学習データが少ないことで検証を行うデータも少なくなることから極端な正答率が多く見られたと考えられる.
また,話者数の増加に伴って選択特徴量数が増加していることがわかる.
これは選択前の選択候補となる特徴量が話者数と比例して大きくなるためである.
一方で全体に対する選択割合に着目すると特に大きな差はないため,問題はないといえる.
いずれのモデルにおいても評価値に選択特徴量数を組み込むことで同様の正答率でありながら,大量の特徴量削減を行うことができることを確認した.
しかし,正答率と選択特徴量数をそれぞれどの程度重視した評価を行うかは検討の余地が残る.
二点交叉手法や,学習モデルとしてLinear SVC,k近傍法,SVCを用いたモデルなどを構築したが表中のモデルとの大きな差は確認できなかった.

\begin{table}[t]
    \caption{特徴量選択を用いたモデルの学習結果}
    \centering
    \begin{tabular}{|c|c|r|r|}
        \hline
        雰囲気 & 話者数 & 正答率 & 選択特徴量数 \\
        \hline\hline
        \multirow{3}{*}{盛り上がり} & 2 & 0.8000 & 10 \\
        & 3 & 0.7956 & 21 \\
        & 4 & 0.8600 & 45 \\ \hline
        \multirow{3}{*}{真面目さ} & 2 & 0.7752 & 16 \\
        & 3 & 0.7694 & 30 \\
        & 4 & 0.8000 & 38 \\ \hline
        \multirow{3}{*}{明るさ} & 2 & 0.8305 & 12 \\
        & 3 & 0.8639 & 33 \\
        & 4 & 0.7000 & 40 \\ \hline
        \multirow{3}{*}{くつろぎ} & 2 & 0.7905 & 13 \\
        & 3 & 0.8167 & 25 \\
        & 4 & 0.9000 & 40 \\ \hline
    \end{tabular}
    \label{tab:learn_result_with_ga}
\end{table}

本研究の対話雰囲気推定モデルは対話単位のデータを入力値としている.
そのため,実際の作業通話にDiscord Botを用いて自動で雰囲気推定を行う場合は,適切なタイミングで話題の切り替わりを判断し,一つのデータとして評価する必要がある.
しかし,本研究で収集する音声データは発話状態時間特徴のみであるため,言語情報を用いた切り替わり判断を行うことができない.
現在は,1 〜 2秒の無音時間を検知した際にデータを切り分けることで,話題の切り替わりを判断する手法を検討している. 


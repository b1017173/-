\chapter{評価実験\label{sec:evaluation_experiment}}
\thispagestyle{plain}

本章では,本研究において構築した対話雰囲気推定モデル及び,それを用いたDiscord Botの評価実験について述べる.

\section{概要}

本研究では対話雰囲気推定モデル及び,Discord Botの評価実験を行う.
雰囲気の一致率やシステムの利便性など主観に基づいた評価を行うため,どちらの評価実験においてもアンケートを用いる.

対話雰囲気推定モデルの評価実験では,モデルの妥当性の評価を目的に,被験者が感じる対話雰囲気とモデルの推定結果がどの程度一致しているかを調査する.
被験者は一度でも作業通話に参加したことがある人とする.
これは作業通話の雰囲気を過去の開催記憶と照らし合わせながら評価をしてもらうことでより説得力のある評価とするためである.
手順はまず本研究の主旨や用語の説明をし,いくつかの対話を聞いてもらいアンケート調査を行う.
調査項目としては「対話を聞いてどのような雰囲気を感じたか」や「対話雰囲気推定モデルの推定結果についてどのように感じたか」などを検討している.
それらの回答をもとに,対話雰囲気推定モデルの推定結果の妥当性を評価する.

Discord Botの評価実験では,Botの利便性の評価を目的に調査を行う.
被験者は対話雰囲気推定モデルの評価実験同様,日常的に作業通話を開催している人とする.
手順はまずBotの利用方法を説明したのちに,Botを交えた短時間の作業通話を行い,出力結果や使用感についてのアンケート実施する.
調査項目としては「Bot操作や表示方法で不明点や不便な点がなかったか」や「雰囲気の修正をしたいと感じたか」などを検討している.
それらの回答をもとに,Discord Botの利便性を評価する.

\chapter{関連研究\label{sec:related_researchs}}
\thispagestyle{plain}

本章では,その上で本研究に関する研究について述べる.
はじめに,対話雰囲気推定を行っている研究について述べる.
次に,SNS上で行われるコミュニケーションを支援している研究について述べる.

\section{対話雰囲気推定に関する研究}
近年,人間と適切な会話を行う対話エージェントを開発するために,対話の雰囲気を推定する研究が数多く報告されている.

人間同士の対話を対象とし,対話雰囲気推定を行った研究として徳久ら\cite{Tokuhisa}の研究がある.
徳久らは対話内容に対してDialogue Act (DA)とRhetorical Relation (RR)を用いた発話分類を行い,対話の盛り上がりとの関係を分析している.
DAとは,各対話内容が「質問」や「要求」,「提案」などのどのようなカテゴリに属するかをラベル付けする手法である.
DAが発話にラベル付けする手法に対し,RRは修辞構造を表す手法である.
結果として,言及される内容の主観性や,「嬉しい」や「つらい」などの感情に関する発話が盛り上がりとの関係が深いことが明らかになっている.
しかし,言語情報を実際の対話から抽出し,会話内容に基づいたラベル付けをし利用することは難しいことも示唆された.

仕草や表情等の非言語情報を用いた対話雰囲気の推定を行った研究として伊藤ら\cite{Ito}の研究がある.
伊藤らは,二者間の対話における盛り上がりに関与する要素を話者の動作や発話から分析している.
結果として,対話の盛り上がりを判断する要素として,発話の重なりやうなずき,笑い等の要素の重要度が高いことが明らかになっている.
また,頭部の動作特徴が場の雰囲気表現に有効であることが示唆された.
しかし,表情認識や動作認識は高いコストを必要とするため,リアルアイムの人間の会話に介入することは難しいという問題がある.
また,遠隔で行われる対話においても,これらの非言語情報が失われるため推定が困難である.

容易に抽出可能な特徴を用いて,対話の雰囲気を推定する研究として豊田ら\cite{Toyota}の研究がある.
豊田らは,発話時間特徴と主観評価を用いた機械学習によって対話雰囲気の推定を試みている.
対話の雰囲気を推定する学習済みモデル(以下,「対話雰囲気推定モデル」)はTree-Augmented Naive Bayes分類器(TAN)と遺伝的アルゴリズムを用いた特徴量選択が採用されている.
モデル構築の過程では,二者の発話全体における同時発話時間の占有率が比較的高い場合,対話が盛り上がっていると推定される傾向があることが明らかになっている.
また,対話をリードしている話者の発話の分散に対して,追従している話者の発話の分散が比較的高い場合にも,対話が盛り上がっていると推定されることが示されている.
構築した対話雰囲気推定モデルを5分割検証法によって分析した結果,「盛り上がり」,「真面目さ」,「親密さ」の推定において80\%を超える全体正答率が示されている.
豊田らの手法は,遠隔で行う対話の雰囲気推定も容易に行える反面,二者対話のみを想定しているため複数人での対話における対話雰囲気推定モデルの精度が確認されていない.
一方で,十分な精度で対話雰囲気の推定を行えていることから,本研究では豊田らのモデル構築手法を参考に対話雰囲気モデルの構築を行う.

\section{SNS上で行われるコミュニケーションに関する研究}

SNS上におけるコミュニティの形成を支援する研究として北村ら\cite{Kitamura}や田村ら\cite{Tamura},久米ら\cite{Kume}の研究がある.
北村らは,関与ベースに基づくTwitterユーザの推薦を行い手法の有効性を示している.
田村らは,Twitterユーザにおける他者とのコミュニケーション量に基づいた会話しやすいユーザの推薦手法を提案している.
久米らは,コンテンツベースの推薦手法にカテゴリ情報を付加し,特徴語抽出手法であるTF-IDF法に各カテゴリへの興味度合いを示す「興味領域」という指標を加えたユーザ推薦手法を提案している.
これらの手法は,ユーザと相性の良い他のユーザを見つける上では有用だが,実際に繋がりを構築する上でユーザは推薦されたユーザに対して接触を図る必要があり,その際の心理的負担は考慮されていない.

SNS上での募集行為を支援する研究として樋川\cite{Hikawa}の研究がある.
樋川は,LINE \cite{LINE}上で実験協力者の募集を,Botを通して行うことができるシステムの構築・評価を行っている.
Botを用いることで適切な距離感の被験者候補に対して募集を行うことができたほか,希薄な関係の人同士におけるシステムを介したコミュニケーションの有効性が示唆された.

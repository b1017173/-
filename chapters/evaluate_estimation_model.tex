\chapter{対話雰囲気推定モデルの評価と考察\label{sec:evaluate_estimation_model}}
\thispagestyle{plain}


\section{対話雰囲気モデルの評価と考察}

% 第\ref{node:develop_estimation_model}節で述べた手法に基づいて対話雰囲気推定モデルを構築した結果を表\ref{tab:learn_result_with_ga}に示す.
表\ref{tab:learn_result_with_ga}には各雰囲気について各話者数のモデルの学習結果の一例を掲載している.
表\ref{tab:learn_result_with_ga}に掲載している学習結果のモデルは全て特徴量選択を行ったモデルである.
表中のモデルは表\ref{tab:ga_setting}のうち評価は選択特徴量数と正答率の重み付け和による評価を採用し,突然変異率は個体突然変異率0.05,遺伝子突然変異率0.10,交叉方法には一様交叉を採用したモデルである.
評価における重み付け$W$は0.50に設定している.
また,学習アルゴリズムにはNaïve Bayesを採用している.
特徴量選択を行う前のモデルの正答率は0.7462であり,特徴量数は144個であった(「盛り上がり」を対象とした話者数3名のモデル).
同雰囲気,話者数を対象としたモデルと比較すると,特徴量選択を行うことで正答率の向上と選択特徴量数の減少が確認できる(正答率0.7956,選択特徴量数21).
加えて,雰囲気や話者数の違いによって精度に大きな差が出ていないことがわかる.
しかし,何度か学習を繰り返すと話者数の増加に伴って正答率が0.40や0.98など極端な数値になることがある.
これは学習データ数が少ないことが原因と考えられる.
例えば,検証に利用できるデータ数が2つしかない場合は正答率が0.0,0.5,1.0の3つしか取り得ない.
このように学習データが少ないことで検証を行うデータも少なくなることから極端な正答率が多く見られたと考えられる.
また,話者数の増加に伴って選択特徴量数が増加していることがわかる.
これは選択前の選択候補となる特徴量が話者数と比例して大きくなるためである.
一方で全体に対する選択割合に着目すると特に大きな差はないため,問題はないといえる.
いずれのモデルにおいても評価値に選択特徴量数を組み込むことで同様の正答率でありながら,大量の特徴量削減を行うことができることを確認した.
しかし,正答率と選択特徴量数をそれぞれどの程度重視した評価を行うかは検討の余地が残る.
二点交叉手法や,学習モデルとしてLinear SVC,k近傍法,SVCを用いたモデルなどを構築したが表中のモデルとの大きな差は確認できなかった.

\begin{table}[t]
    \caption{特徴量選択を用いたモデルの学習結果}
    \centering
    \begin{tabular}{|c|c|r|r|}
        \hline
        雰囲気 & 話者数 & 正答率 & 選択特徴量数 \\
        \hline\hline
        \multirow{3}{*}{盛り上がり} & 2 & 0.8000 & 10 \\
        & 3 & 0.7956 & 21 \\
        & 4 & 0.8600 & 45 \\ \hline
        \multirow{3}{*}{真面目さ} & 2 & 0.7752 & 16 \\
        & 3 & 0.7694 & 30 \\
        & 4 & 0.8000 & 38 \\ \hline
        \multirow{3}{*}{明るさ} & 2 & 0.8305 & 12 \\
        & 3 & 0.8639 & 33 \\
        & 4 & 0.7000 & 40 \\ \hline
        \multirow{3}{*}{くつろぎ} & 2 & 0.7905 & 13 \\
        & 3 & 0.8167 & 25 \\
        & 4 & 0.9000 & 40 \\ \hline
    \end{tabular}
    \label{tab:learn_result_with_ga}
\end{table}

本研究の対話雰囲気推定モデルは対話単位のデータを入力値としている.
そのため,実際の作業通話にDiscord Botを用いて自動で雰囲気推定を行う場合は,適切なタイミングで話題の切り替わりを判断し,一つのデータとして評価する必要がある.
しかし,本研究で収集する音声データは発話状態時間特徴のみであるため,言語情報を用いた切り替わり判断を行うことができない.
現在は,1 〜 2秒の無音時間を検知した際にデータを切り分けることで,話題の切り替わりを判断する手法を検討している. 



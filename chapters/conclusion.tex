\chapter{結論と今後の展望\label{sec:conclusion}}
\thispagestyle{plain}

\section{結論}

本研究の目的は,作業嗜好の合う無縁ユーザの気軽なマッチングを実現し,より効果的な作業通話の開催機会の増加を目指すことである.
その実現方法として,作業通話における参加者のマッチングを支援するシステムを提案した.
システムの機能として作業嗜好の可視化機能と,募集文の生成機能を提案した.
特に作業嗜好の可視化を効果的に行う要素として対話雰囲気に着目した.

加えて,対話雰囲気を推定する対話雰囲気推定モデルの構築を続けて行った.
本研究で構築した対話雰囲気推定モデルは2 〜 4名を対象に「盛り上がり」「真面目さ」「明るさ」「くつろぎ」の推定を行うモデルであり,0.8程度のモデル正答率を確認できた.

\section{今後の展望}

対話雰囲気推定モデルの構築に関する今後の課題として,より効果的な個体評価手法の検討や学習データの拡充が明らかとなった.
特に,学習データの拡充が行うことでより学習効率の良い対話雰囲気推定モデル構築手法の発見や,新たな課題の発見が期待できるため早急な解決が求められている.
また,本研究で提案したシステムは提案に留まっており,構築と検証も求められている.
その際には,より多くの人が利用しやすいUIやUXも踏まえてシステムを検討していく必要がある.

将来的には作業通話だけでなく,オンラインゲームの通話を始めとした複数人による通話全般の活動雰囲気の推定等に応用できる対話雰囲気推定モデルやシステムの構築を期待している.

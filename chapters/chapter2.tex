\chapter{アプローチ\label{sec:approach}}
\thispagestyle{plain}

本章では第\ref{sec:introduction}章で述べた問題に対する本研究のアプローチについて述べる.
第\ref{sec:introduction}章では以下の問題について述べた.

\begin{enumerate}[i.]
    \item 大量の募集の中からどれが自身の作業嗜好に合うものであるか判断することが困難
    \item 募集に対して参加の意思表明をすることの心理的負担が大きい
\end{enumerate}

\section{大量の募集の中からどれが自身の作業嗜好に合うものであるか判断することが困難であることに対するアプローチ}

本問題では作業嗜好の可視化が有効である.
可視化を行う項目,つまり対象とする作業嗜好の要素には作業時間帯や作業内容などいくつかの候補が存在するが,本研究では対話雰囲気に着目する.
本研究では,対話雰囲気を「盛り上がっている」,「真面目である」など複数人からなる対話における場の空気感と定義する.
対話雰囲気は参加者の心情や,意欲から形成されることが多く,作業の進め方に強く影響を与える.本研究では対話雰囲気が作業嗜好の大きな要素になると仮定し,対話雰囲気を推定し可視化するシステムの構築を行う.

\section{募集に対して参加の意思表明をすることの心理的負担が大きいことに対するアプローチ}

本問題の解決には,無縁ユーザに向けた募集の枠組みの構築に着目する.
そもそも本問題の根本的な原因として,無縁ユーザとの作業通話という文化が十分に浸透していないことが挙げられる.
加えて,多くの募集が「今夜作業通話したいな」等,SNS募集とコミュニティ募集のどちらなのかを一目で判断できないことが挙げられる.
そこで無縁ユーザに向けた募集の枠組みを設ける.具体的には無縁ユーザに向けた投稿文(ツイート)を生成できるシステムの提案を行う.
これによって,募集対象者が自身に向けて募集が行われていることや,そのような募集方法が存在することを認知する機会が増える.
このようにシステムを通したコミュニケーションは受け入れやすいということが報告されている\cite{Harada}\cite{Kimura}\cite{Nishimura}\cite{Tsuzuki}.
これらを踏まえ募集対象者の参加に対する心理的負担の低下を目指す.
作成する募集の投稿には前述した対話雰囲気を掲載することで,募集対象者は募集者の活動傾向を推察できる.
これによりさらなる心理的負担の軽減を目指す.

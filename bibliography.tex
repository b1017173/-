% TODO: 参考文献を以下のように記入.

\begin{thebibliography}{99}
\bibitem{Discord}
``Discord'': https://discord.com/, (参照 2023-01-12).
\bibitem{Zajonc}
Zajonc, R, B.: Social facilitation, Science, Vol.149, pp.269-274, 1965.
\bibitem{Matsumoto}
松本芳之:観察者効果に関するフィード研究:技能水準,課題難度,遂行状況の関係について,実験社会心理学研究,Vol.26,No.2,pp.115-123,1987.
\bibitem{Miyamoto}
宮本正一:観察者の存在による選択反応時間の抑制,心理学研究,Vol.58,No.4,pp.240-246,1987.
\bibitem{Twitter}
``Twitter'': https://twitter.com/, (参照 2023-01-12).
\bibitem{Harada}
原田悦子:人の視点から見た人工物研究(認知科学モノグラフ6),共立出版,1997.
\bibitem{Kimura}
木村泰之,都築誉史:集団意思決定とコミュニケーション・モードコンピュー夕・コミュニケーション条件と対面コミュニケーション条件の差異に関する実験社会心理学的検討,実験社会心理学研究,Vol.38,pp.183-192,1998.
\bibitem{Nishimura}
西村洋一:対人不安、インターネット利用、およびインターネットにおける対人関係,社会心理学研究,Vol.19,pp.124-134,2003.
\bibitem{Tsuzuki}
都築誉史,木村泰之:大学生におけるメディア・コミュニケーションの心理的特性に関する分析一対面、携帯電話、携帯メール、電子メール条件の比較,立教大学応用社会学研究,Vol.42,pp.15-24,2000.
\bibitem{Tokuhisa}
徳久良子,寺嶌立太:雑談における発話のやりとりと盛り上がりの関連,人工知能学会論文誌,Vol.21,No.2A,pp.133-142,2006.
\bibitem{Ito}
伊藤秀樹,重野真也,西本卓也,荒木雅弘,新美康永:対話における雰囲気の分析,情報処理学会研究報告 SLP,音声情報処理,Vol.2002,No.10,pp.103-108,2002.
\bibitem{Toyota}
豊田薫,宮越喜浩,山西良典,加藤昇平:発話状態時間長に着目した対話雰囲気推定,人工知能学会論文誌,Vol.27,No.2SP-B,pp.16-21,2012.
\bibitem{Kitamura}
北村太一,小川祐樹,諏訪博彦,太田敏澄:コミュニケーションに着目したTwitterフォローユーザ推薦,人工知能学会全国大会論文集,Vol.26,pp.1-4,2012.
\bibitem{Tamura}
田村政人,小林亜樹:Twitterにおける会話しやすいユーザの推薦手法,情報処理学会第75回全国大会講演論文集,Vol.2013,No.1,pp.605-607,2013.
\bibitem{Kume}
久米雄介,打矢隆弘,内匠逸:興味領域を考慮したTwitterユーザ推薦手法の提案と評価,情報処理学会研報告 SIG,Vol.2015-ICS-179,No.1,pp.1-8,2015.
\bibitem{Hikawa}
樋川一幸:適切な距離の学生に依頼可能なBotを利用した実験協力者募集手法の研究,明治大学大学院先端数理科学研究科修士(工学),2020. 
\bibitem{LINE}
``LINE'': https://line.me/ja/, (参照 2023-01-12).
\bibitem{PASD}
NII音声資源コンソーシアム:対話音声コーパス(PASD),1996.
\end{thebibliography}

% TODO: 参考文献を以下のように記入.

\begin{thebibliography}{99}
 \bibitem{Discord}
“Discord”. https://discord.com/, (参照 2023-01-12).
 \bibitem{Zajonc}
Zajonc.R.B, Social facilitation, Science, 149, p.269-274, 1965.
\bibitem{Matsumoto}
松本芳之, 観察者効果に関するフィード研究:技能水準, 課題難度, 遂行状況の関係について, 実験社会心理学研究, 第26巻, 第2号, p.115-123, 1987.
\bibitem{Miyamoto}
宮本正一, 観察者の存在による選択反応時間の抑制, 心理学研究, 第58巻, 第4号, p.240-246, 1987.
\bibitem{Twitter}
“Twitter”. https://twitter.com/, (参照 2023-01-12).
\bibitem{Harada}
原田悦子, 人の視点から見た人工物研究(認知科学モノグラフ6), 共立出版, 1997.
\bibitem{Kimura}
木村泰之・都築誉史, 集団意思決定とコミュニケーション・モー ドコンピュー 夕・ コミュニケーション条件と対面コミュニケーション条件の差異に関する実験社会心理学的検討, 実験社会心理学研究, 38巻, p.183-192, 1998.
\bibitem{Nishimura}
西村洋一, 対人不安、 インターネット利用、およびインターネットにおける対人関係, 社会心理学研究, 19巻, p.124-134, 2003.
\bibitem{Tsuzuki}
都築誉史・木村泰之, 大学生におけるメディア・ コミュニケーションの心理的特性に関する分析一対面、携帯電話、携帯メール、電子メール条件の比較, 立教大学応用社会学研究, 42巻, p.15-24, 2000.
\bibitem{Tokuhisa}
徳久良子・寺嶌立太, 雑談における発話のやりとりと盛り上がりの関連, 人工知能学会論文誌, 21巻, 2号A, p.133-142, 2006.
\bibitem{Ito}
伊藤秀樹・重野真也・西本卓也・荒木雅弘・新美康永, 対話における雰囲気の分析, 情報処理学会研究報告 SLP, 音声情報処理, No.10, p.103-108, 2002.
\bibitem{Toyota}
豊田薫・宮越喜浩・山西良典・加藤昇平, 発話状態時間長に着目した対話雰囲気推定, 人工知能学会論文誌, 27巻, 2号 SP-B, p.16-21, 2012.
\bibitem{Kitamura}
北村太一・小川祐樹・諏訪博彦・太田敏澄, コミュニケーションに着目したTwitterフォローユーザ推薦, 人工知能学会全国大会論文集, 3E1-R-6-5, p.1-4, 2012.
\bibitem{Tamura}
田村政人・小林亜樹, Twitterにおける会話しやすいユーザの推薦手法, 情報処理学会第75回全国大会講演論文集, 2013(1), p.605-607, 2013.
\bibitem{Kume}
久米雄介・打矢隆弘・内匠逸, 興味領域を考慮したTwitterユーザ推薦手法の提案と評価, 情報処理学会研報告 SIG, Vol.2015-ICS-179, No.1, p.1-8, 2015.
\bibitem{Hikawa}
樋川一幸, 適切な距離の学生に依頼可能なBotを利用した実験協力者募集手法の研究, 明治大学大学院先端数理科学研究科修士(工学), 2020. 
\bibitem{LINE}
“LINE”, https://line.me/ja/, (参照 2023-01-12).
\bibitem{PASD}
NII 音声資源コンソーシアム:対話音声コーパス(PASD)(1993-1996).
\end{thebibliography}
